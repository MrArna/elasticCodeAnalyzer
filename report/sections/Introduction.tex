\section{Introduction}
\label{sec:introduction}
Versioning systems are one milestone in the field of software management and development especially for huge project and leading companies. In this direction several providers of this kind of service raised, providing for huge companies, but also for single or small group of developers, a simple and quick way to access those services via Internet. These providers also expose APIs in order to allow third parties to easily and automatically access information about the versioned project they own.
One of those, it's GitHub and with our work we would like to prove that those APIs can be used for more higher purposes than retrieving information and downloading public accessible repositories. Maybe integrating and analysing the huge amount of retrievable information it is possible to find pattern in the developer and committers behaviour. This could be helpful in order to educate developers, but also to auto-detect common semantic mistakes they are used to make. So, the final goal of our project is to prove that aggregating information provided by GitHub about its versioned repositories it could be possible to mining from them useful information about programmers behaviour, in particular about the common they are used to do and trying to cluster them into families.

The rest of the report is organized as follows: \ref{sec:implementation} presents the proposed approach and some implementation details; preliminary results are detailed in Section \ref{sec:results}, discussing the limitations of this work in Section~\ref{sec:limitations}, finally drawing some conclusions in Section~\ref{sec:conclusion}.